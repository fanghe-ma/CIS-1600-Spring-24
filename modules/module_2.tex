\chapter{Module 2}

\section{Counting subsets}

\subsection{Cardinality of powerset}
The cardinality of the powerset $2^A$ is the number of subsets of  $A$
\[
   |2^A| = 2^{|A|}
\] 


\section{Permutations}

\begin{framed}
   \textbf{Definition}: Let  $A$ be a non-empty set with $n$ elements 
   \[
     |A| = n
   \] 
   A \textbf{permutation} of $A$ is an ordering of all elements of $A$ without repetition \\
   \[
     n!
   \] 

   A partial permutation of $r$ out of $n$ elements of $A$ consists of picking $r$ of the elements and ordering them without repetition
   \[
     \frac{n!}{(n -r)!}
   \] 
\end{framed}

\section{Logical structure of statements}

\subsection{Logical connectives}
\begin{framed}
   Connectives
\begin{itemize}
   \item \textbf{conjunction}:  $\quad "and" \quad \land$
   \item \textbf{disjunction}:  $\quad "or" \quad \lor$
   \item \textbf{implication}:  $\quad "if-then" \quad \Rightarrow$
   \item \textbf{negation}:  $\quad "not" \quad \lnot$
\end{itemize}
\end{framed}

For example, a letter $l$ is either a vowel or a consonant can be expressed as
\[
   letter(l) \Rightarrow [
   (vowel(l) \land \lnot consonant(l)) \lor (consonant(l) \land \lnot vowel(l))
   ] 
\] 
\[
   letter(l) \Rightarrow [
   vowel(l) \land consonant(l)) \land \lnot (vowel(l) \land consonant(l))
   ]
\] 

\subsection{Logic set-builder notation}
\begin{framed}
   If $P(x)$ is some logical statement about $x$, $A$ is the set of elements that satisfies the logical statement
   \[
     A = \{ x | P(x)\} 
   \] 
  
\end{framed}


\section{Implication, conditional and equivalence}
\begin{framed}
\begin{itemize}
   \item \textbf{implication}: if $P_1$ then $P_2$ : $P_1 \Rightarrow P_2$ 
   \item \textbf{biconditional / equivalence}: if $P_1$ then $P_2$ and if  $P_2$ then $P_1$ 
      \[
         \left( P_1 \Rightarrow P_2 \right)  \lor \left( P_2 \Rightarrow P_1 \right) 
      \] 
      \[
        P_1 \iff P_2
      \]
\end{itemize}
\end{framed}

\section{Symmetric difference}
\begin{framed}
   The symmetric difference of two sets $A, B$ is 
   \[
     A \triangle B = \left( A \setminus B \right) \cup \left( B \setminus \right) 
   \] 
   \[
     A \triangle B = \{ x | \left( x \in A \land x \notin B \right) \lor \left( x \notin A \land x \in B \right)   \} 
   \] 
\end{framed}


\section{Two basic proof patters}

\subsection{Proof pattern for implication}

\begin{framed}
   To prove that
   \[
     P_1 \Rightarrow P_2
   \] 
   \textbf{Proof pattern}
   \begin{enumerate}
      \item assert premise $P_1$
      \item logical / mathematical consequences
      \item assert conclusion $P_2$
   \end{enumerate}
\end{framed}

\subsection{Proof pattern by cases}

\begin{framed}
   Assuming $P_1 \lor P_2$, to prove $P_3$

   \textbf{Proof pattern} \\

   Assert $P_1 \lor P_2$ \\

   \textbf{Case 1}: Assert $P_1$
   \begin{itemize}
      \item logical / mathematical consequences 
      \item assert $P_3$
   \end{itemize}

   \textbf{Case 2}: Assert $P_2$
   \begin{itemize}
      \item logical / mathematical consequences 
      \item assert $P_3$
   \end{itemize}

   $P_3$ proven assuming $P_1 \lor P_2$
\end{framed}

\section{Proofs regarding sets}
\begin{framed}
   To show $A = B$ for sets $A, B$, show that
   \begin{itemize}
      \item $A \subseteq B$
      \item $B \subseteq A$
   \end{itemize}
   
  
\end{framed}


\section{Combinations}

\begin{framed}
   Let $A$ be a non-empty set with $n$ elements, and $r$ be a natural number \\

   A combination of $r$ elements from the $n$ elements of $A$ is an \textbf{unordered} selection of $r$ of the $n$ elements of $A$ \\

   A combination is the same as a \textbf{subset} $S$ of $A$ of size $r$ \\

   The number of combinations is given by 
   \[
     \begin{pmatrix} n \\ r \end{pmatrix}  = \frac{n!}{r! (n-r)!}
   \] 
\end{framed}

\section{Predicates and Quantifiers}

\subsection{Predicates}
\begin{framed}
   \textbf{Predicates} are undetermined logical statements because they contain variables whose values are not specified, e.g.
   \[
     odd(p), vowel(l)
   \] 
\end{framed}

\subsection{Quantifiers}
\begin{framed}
   \textbf{Quantifiers} include 
   \begin{itemize}
      \item universal quantifiers, $\forall x$
      \item existential quantifiers, $\exists x$
   \end{itemize}
\end{framed}

For all positive integers there is a strictly bigger positive integer that is a prime
\[
   \forall x \in \mathbb{Z}^{+} \exists  y \in \mathbb{Z}^+ (y > x) \land prime(y)
\] 




