\chapter{Graph Theory I}

\section{Terminology}
\begin{framed}
   A undirected graph is a pair 
   \[
     G = (V, E)
   \] 

   Where $V$ is a finite non-empty set of vertices \\

   $E \subseq 2^V$ is a finite and possibly empty set of edges. 
\end{framed}

\subsection{Vertex degree}
\begin{framed}
   \textbf{Definition}: The degree of a vertex, $deg(u)$ is the number of neighbors of $u$. 
\end{framed}

\begin{framed}
   \textbf{Handshaking Lemma}: the sum of degrees of all nodes in a graph is twice the number of edges
   \[
      \sum_{v \in V} deg(v) = 2 |E|
   \] 
\end{framed}

\begin{framed}
   \textbf{Proposition}: In any graph, there are an even number of vertices of odd degree \\


   \textbf{Proof}: Consider $V = V_e \cup V_o$, $V_e \cap V_o = \emptyset$ . 

   \[
      \sum_{v \in V} deg(v) = \sum_{v \in V_e} deg(v) + \sum_{v \in V_o} deg(v)
   \] 
  
\end{framed}

\section{Special Graphs}

\subsection{Edgeless}
\begin{framed}
   An \textbf{edgeless } graph is $G = (V, E)$ where
   \[
     E = \emptyset
   \] 
   \[
     |E| = 0
   \] 
\end{framed}


\subsection{Complete}

\begin{framed}
   A \textbf{complete} graph is 
   \[
     K_n = (V, E) \text{ where } n \geq 1
   \] 
   \[
     |V| = n
   \] 
   \[
     |E| = \begin{pmatrix} n //2 \end{pmatrix} 
   \] 
  
\end{framed}

\subsection{Paths}
\begin{framed}
   A path graph is denoted  
   \[
     P_n = (V, E)
   \] 

   where
   \[
     |V| = n
   \] 
   Note that for a path graph, 
   \[
     |E| = n - 1
   \] 

   Note that for $ n \geq 3$ 
   \[
     \left| \{ v \in V : deg(v) = 1 \}  \right|  = 2
   \] 
   \[
     \left| \{ v \in V : deg(v) = 2 \}  \right|  = n - 2
   \] 
  
\end{framed}


\subsection{Cycles}
\begin{framed}
   A cycle is denoted
   \[
     C_n = (V, E) \text{ for } n \geq 3
   \] 


   Note that
   \[
     \forall v \in V, deg(v) = 2
   \] 
   \[
     C_n \implies \text{ all vertices have degree two}
   \] 
   but the converse is not true
\end{framed}

\subsection{Grids}
\begin{framed}
   A grid has $m$ rows and $n$ columns \\

   When  $m, n \geq 3$, 
   \[
    \left| \{ v \in V : deg(v) = 2 \}  \right|  = 4
   \] 
   \[
    \left| \{ v \in V : deg(v) = 3 \lor deg(v) = 4\}  \right|  = m \times n - 4
   \] 

   Note that the number of edges 
   \[
      |E_{horizontal} | = m(n-1)
   \] 
   \[
      |E_{vertial} | = n(m-1)
   \] 

   \[
     |E| = m(n-1) + n(m-1) 2mn -m - n
   \] 
\end{framed}

\section{Walks and paths} 

\begin{framed}
   \textbf{Definition}: A walk is a non-empty sequence of vertices consecutively linked by edges
   \[
     u_0, u_1 \hdots u_k
   \] 

   such that
   \[
      u_0 - u_1 - \hdots u_k
   \] 

   We say that 
   \begin{itemize}
      \item $u_0, u_k$ are the endpoints of the walk
      \item $u_0, u_k$ are connected by this walk
      \item the length of the walk is $k$ (there are $k + 1$ nodes)
   \end{itemize}

   Note that a single vertex is a walk of length $0$
\end{framed}

\begin{framed}
   \textbf{Definition}: A path is a walk with all vertices distinct. 
\end{framed}

\begin{framed}
   \textbf{Proposition}: When there is a walk from $u_0$ to $u_n$ of length $n \geq 3$, and  $u_0 \neq u_n$, then there exists vertices $v_1 \hdots v_m$ such that $u_0 - v_1 - \hdots v_m - u_n$ is a path of length $\leq n$ \\

   Alternatively, there exists a path whose sequence of nodes and edges are a subsequence of the sequence of nodes and edges in the walk. 
\end{framed}


\section{Connected Components}

\begin{framed}
   \textbf{Definition}: two vertices $u, v $ of graph $G = (V, E)$ are said to be \textbf{connected} if there exists some walk with endpoints $u, v$. \\

   Note that connectivity is
   \begin{itemize}
      \item transitive
      \item reflexive
      \item symmetric
   \end{itemize}
\end{framed}

\begin{framed}
   A \textbf{CC} of $G = (V, E)$ is a set of vertices $C \subseteq V$ such that
   \begin{itemize}
      \item any two vertices in $C$ are connected 
      \item there is no strictly bigger set of vertices $C \subset C' \subseteq V$ such that any two vertices in $C'$ are also connected 
   \end{itemize}
   $C$  is a maximally connected set of vertices in $G$
\end{framed}

\begin{framed}
   \textbf{Proposition}: any two distinct CCs are disjoint
\end{framed}

\begin{framed}
   \textbf{Definition}: A \textbf{connected graph} is a graph with exactly one connected component
   \[
     |CC| = 1
   \] 
  
\end{framed}

\section{Counting CCs}
\begin{framed}
   \textbf{Proposition}: For any graph $G = (V, E)$
   \[
     |V| - |E| \leq |CC| \leq |V|
   \] 
\end{framed}

\begin{framed}
   \textbf{Proposition}: For any graph $G = (V, E)$
    \[
     |V| - |CC| \leq |E| \leq \begin{pmatrix} n \\2 \end{pmatrix} 
   \] 
\end{framed}






  















