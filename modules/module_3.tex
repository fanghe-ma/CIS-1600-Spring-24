\chapter{Module 3}

\section{Stars and Bars}

Stars and bars are used to count the number of ways to distribute $n$ \textbf{indistinguishable} objects into $r$ distinguishable categories, given by
\[
  \begin{pmatrix} 
    n + r - 1 \\ r - 1  
  \end{pmatrix}
\] 

\section{Negating statements}

To disprove a statement $P$ is to prove its negation $\lnot P$ 

\subsection{Negation of disjunction / conjunction}

\textbf{IMPT!!}
\[
  \text{the negation of disjunction is conjunction}
\] 
\[
  \lnot( P_1 \land P_2) = (\lnot P_1) \lor (\lnot P_2)
\] 


\[
  \text{the negation of conjunction is disjunction}
\] 
\[
  \lnot( P_1 \lor P_2) = (\lnot P_1) \land (\lnot P_2)
\] 

\subsection{Negation of quantifiers}

\[
  \text{the negation of universal is existential}
\] 
\[
  \lnot(\forall x P(x)) = \exists x \lnot P(x)
\] 

\[
  \text{the negation of existential is universal}
\] 
\[
  \lnot(\exists x P(x)) = \forall x \lnot P(x)
\] 

\subsection{Negation of implication}
\[
  \text{The negation of "if premise then conclusion" is premise and the negation of the conclusion}
\] 
\[
  \lnot(P_1 \Rightarrow P_2) = P_1 \land \lnot P_2
\] 

\section{Converse and Contrapositive}

Given an implication
\[
   P_1 \Rightarrow P_2 \text{ (if $P_1$ then $P_2$)}
\] 

The \textbf{converse} is
\[
  P_2 \Rightarrow P_1 \text{ (if $P_2$ then $P_1$)}
\] 

The \textbf{contrapositive} is 
\[
  \lnot P_2 \Rightarrow \lnot P_1 \text{ (if not $P_2$ then not $P_1$)}
\] 

\textbf{NOTE} that the contrapositive is \textbf{logically equivalent} to the original implication

\section{Truth tables}

\begin{center}
   \begin{tabular}{| c | c | c | c | c |}
      $p$ & $q$ & $p \land q$ & $p \lor q$ & $p \Rightarrow q$ \\
      T & T & T & T & T \\
      T & F & F & T & F \\
      F & T & F & T & T \\
      F & F & F & F & T
   \end{tabular}
\end{center}


\section{Vacuous implications}

Note that is the premise is false then the implication is true, regardless of the conclusion. \\ 

False implies everything \\

Implications where premise is always false is said to \textbf{hold vacuously}

\section{Prove by contrapositive}

Instead of proving 
 \[
  p \Rightarrow q
\] 

we prove
\[
  \lnot q \Rightarow \lnot p
\] 

\section{Logical equivalence}

Two boolean expressions are \textbf{logically equivalent} if they yield the same truth value for the same truth assignments to their variables

\begin{align*}
   p \Rightarrow q & \equiv \lnot q \Rightarrow \lnot p \quad \text{ Contrapositive} \\
   p \lor q \Rightarrow r &\equiv (p \Rightarrow r) \land (q \Rightarrow r) \quad \text{ By-cases} \\
   p \Rightarrow q & \equiv \lnot p \lor q \quad\text{ Law of Implication} \\
   \lnot (p \Rightarrow q) & \equiv p \land \lnot q \quad\text{ Disproving implication} \\
   \lnot (p \lor q) & \equiv \lnot p \land \lnot q \quad\text{ De Morgan's Law I} \\
   \lnot (p \land q) & \equiv \lnot p \lor \lnot q \quad\text{ De Morgan's Law II} \\
   \lnot \lnot p & \equiv p \quad \text{ Law of Double Negation}
\end{align*}

\section{Proofs by contradiction}

\subsection{Proofs by contradiction I}
\begin{framed}
A statement $P$ and  $\lnot P$ is called a contradiction, and is always false. \\ 

To prove $P$, we prove instead that  if (not $P$ ) then $C$, where $C $ is a contradiction.
\end{framed}

\subsection{Proofs by contradiction II}

\begin{framed}
To prove if $P$ then $Q$, we can instead prove if  $P$ and not $Q$ then $C$
\[
  p \Rightarrow q \equiv p \land \lnot q \Rightarrow F
\] 
\end{framed}
\section{Permutations}


The number of permutations of a multiset is
\[
   \frac{n_{total}!}{n_1! n_2! \hdots n_k!}
\] 



