\chapter{Module 13}

\section{Spanning Trees}

\begin{framed}
   \textbf{Definition}: A spanning subgraph of $G = (V, E)$ is a subgraph whose vertex set is $V$. \\

   A spanning tree of a connected graph $G$ is a spanning subgraph that is a tree. \\

   A spanning forest of a graph $G$ consist of a spanning tree for each CC of $G$. 
\end{framed}

\subsection{Existence of spanning trees}

\begin{framed}
   \textbf{Proposition}: Every connected graph has a spanning tree. 
\end{framed}

\begin{framed}
   \textbf{Proposition}: Removing a cut edge increases the number of connected components by exactly one. 
\end{framed}


\section{Graph coloring}

\subsection{Definitions}

\begin{framed}
   \textbf{Definition}: For $G = (V, E)$ and some $k \in \mathbb{Z}+$, a \textbf{k-coloring} of $G$ is a function $f : V \rightarrow [1 \hdots k]$ 
\end{framed}

\begin{framed}
   \textbf{Definition}: A coloring is \textbf{proper} when 
   \[
     \forall (u-v) \in E, f(u) \neq f(v)
   \] 
\end{framed}

\begin{framed}
   \textbf{Definition}: A graph is k-colorable if it admits a proper k coloring.  k-colorable implies j-colorable for $j > k$. 
\end{framed}

\begin{framed}
   \textbf{Definition}: The smallest $k$ such that $G$ is k-colorable is called the \textbf{chromatic number} of $G$, denoted $\chi(G)$ 
  
\end{framed}


\subsection{Chromatic number of graphs} 
\begin{framed}
   \textbf{Proposition}: Only edgeless graphs are $1$-colorable
   \[
     \chi(G) = 1 \iff E = \emptyset
   \] 
\end{framed}

\begin{framed}
   \textbf{Proposition}: path graphs are 2-colorable
   \[
     \forall n \geq 2, \chi(P_n) = 2
   \] 
\end{framed}

\begin{framed}
   \textbf{Proposition}: Word graphs are 2 or 3 colorable depending on the parity of number of indices 
   \[
     \chi(C_n) = 
     \begin{cases}
        2 \text{ if } even(n) \\
        3 \text{ if } odd(n) \\
     \end{cases}
   \] 
\end{framed}

\begin{framed}
   \textbf{Proposition}: Connected graphs are $n$-colorable. \\
   \[
     \chi(K_n) = n
   \] 
   \textbf{Proof}: prove by contradiction that if  $K_n$ is m-colorable for some $m < n$, then at least 2 vertices have the same color, but the two vertices are adjacent. 
\end{framed}

\subsection{Bipartite graphs}

\begin{framed}
   \textbf{Definition}: 2-colorable graphs are called \textbf{bipartite}. 
  
\end{framed}

\begin{framed}
   \textbf{Proposition}: Every path graph is bipartite.
\end{framed}

\begin{framed}
   \textbf{Proposition}: Every cycle graph with even number of vertices is bipartite.
\end{framed}

\begin{framed}
  \textbf{Proposition}: Every tree is bipartite.  \\

  \textbf{Proof}: By induction on the number of vertices
\end{framed}

\begin{framed}
   \textbf{Proposition}: A graph is \textbf{bipartite} iff it does not contain a cycle of odd length.
\end{framed}

\begin{framed}
   \textbf{Proposition}: If $S$ is a subgraph of $G$ then 
   \[
     \chi(S) \leq \chi(G)
   \] 

   Or, every subgraph of a bipartite graph is bipartite. 
\end{framed}

\subsection{Distance in a connected graph}
\begin{framed}
   \textbf{Definition}: The distance between two vertices $u, v \in V$, denoted $d(u, v)$ is the length of the shortest path from $u$ to $v$. 
   \begin{itemize}
      \item $d(u, u) = 0$
      \item $d(u, v) = d(v, u)$
      \item $d(u, v) \leq d(u, w) + d(w, v)$
   \end{itemize}
\end{framed}


\section{Cliques and independent sets}

\begin{framed}
   \textbf{Definition}: A \textbf{clique} of $G$ denotes the complete subgraph of $G$ or a subset of $V' \subseteq V$ such that it induces a complete subgraph. \\

   A \textbf{clique} is a subset of vertices such that any two are adjacent. 
\end{framed}

\begin{framed}
   \textbf{Definition}: A \textbf{independent set} is a subset of vertices $V' \subseteq V$ such that no two vertices are adjacent. 
\end{framed}

\begin{framed}
   \textbf{Proposition}: A graph is k-colorable if its set of vertices can be partitioned into $k$ independent sets. 
\end{framed}

\begin{framed}
   \textbf{Proposition}: The complement of $G = (V, E)$ is $\overline{G}= (V, \overline{E)}$ where
   \[
     \overline{E} = \{ (u, v) : u, v \in V \land u\neq v \land (u, v) \notin E \} 
   \] 
\end{framed}

\begin{framed}
   \textbf{Proposition}: A subset $V' \subseteq V$ is a clique in $G$ iff it is an independent set in $\overline{G}$. 
\end{framed}











