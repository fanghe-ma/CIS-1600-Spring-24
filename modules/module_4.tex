\chapter{Module 4}

\section{Binomial Theorem}

The binomial coefficient is \[
  \begin{pmatrix} n \\ r \end{pmatrix} 
\] 

The binomial theorem states that for any real $a, b$, and any natural number $n$, 
\[
   (a + b)^n = \begin{pmatrix} n \\ 0 \end{pmatrix} a^n + \begin{pmatrix} n \\1  \end{pmatrix} a^{n-1}b + \hdots \begin{pmatrix} n \\n \end{pmatrix} b^n = \sum_{i = 0}^{n} \begin{pmatrix} n \\ i  \end{pmatrix} a^{n-i}b^i
\] 

\section{Combinatorial proofs}

A combinatorial identity
\[
  \begin{pmatrix} n \\ r \end{pmatrix}  = \begin{pmatrix} n \\ n - r \end{pmatrix} 
\] 

Pascal's identity
\[
  \begin{pmatrix}  n \\ k \end{pmatrix}  = \begin{pmatrix} n - 1 \\k - 1 \end{pmatrix}  + \begin{pmatrix} n - 1 \\k \end{pmatrix} 
\]  

\section{Functions}

\begin{framed}
   \textbf{Definition} A function or a mapping denoted $f : A \rightarrow B$ consists of 
   \begin{itemize}
      \item set $A$, called \textbf{ domain}
      \item set $B$, called \textbf{ codomain}
      \item a mapping, or a way of associating \textbf{every} element of domain with a unique element of the codomain \[
        x \in A, f(x) \in B
      \] 
   \end{itemize}

   The \textbf{range} of a function $f:B \rightarrow B$ is
   \[
     Ran(f) = \{  y | y \in B \land \exists x \in A\ y = f(x) \} 
   \] 

   Note that
   \[
     Ran(f) \subseteq B
   \] 
\end{framed}

\subsection{Set of all functions}

Given two  sets  $A, B$, the set
\[
   \{ f | f: A \rightarrow B \} \text{ is denoted by} B^A
\] 

If $|A| = r$ and $|B| = n$, then the number of different functions with domain $A$ and codomain $B$ is
 \[
    |B^A| = |B|^{|A|} = n^r
\] 

If $|A| = |B| = |C| = n$, how many functions are there with domain  $B^A$ and codomain $2^C$? \\

Such functions map functions in $B^A$ to the superset of $C$

\[
   |2^{C^{B^A}}| = |2^C|^{|B^A|} = 2^{n^{n^n}} = 2^{n^{n+1}}
\] 


\section{Integer intervals}

\begin{framed}
   \textbf{Definition}: An integer interval $[m \hdots n]$ ($m < n$ ) is the set of all integers that lay between $m, n$ inclusive
   \[
      [m \hdots n] = \{  k \in \mathbb{Z} | m \leq m \leq n \} 
   \] 
\end{framed}

\section{Surjections and injections}

\subsection{Surjective functions}

\begin{framed}
   \textbf{Definition}: A function $f: A \rightarrow B$ is surjective if $Ran(f) = B$, or that for every  $y \in B$ there exists $x \in A$ such that $y = f(x) $ \\


   Method of proof: 
   \begin{itemize}
      \item show that for every $y \in B$, there is some $x \in A$ such that $f(x) = y$
   \end{itemize}
\end{framed}

\subsection{Injective functions}

\begin{framed}
   A function $f: A \rightarrow B$ is injective if it maps distinct elements to distinct elements, i.e. 
   \[
     \text{ for every } x_1 \neq x_2 \text{ in the domain we have } f(x_1) \neq f(x_2)
   \]  or
   \[
     \text{ for every }x_1, x_2 \in A, f(x_1) = f(x_2) \implies x_1 = x_2
   \] 

   Method of proof: 
   \begin{itemize}
      \item show that for every $x_1 \neq n_2$ then $f(x_1) \neq f(x_2)$
      \item show that for every $x_1, x_2$, $f(x_1) = f(x_2) \implies x_1 = x_2$
   \end{itemize}
\end{framed}

\subsection{Surjection \& injection rules}

\begin{framed}
   \textbf{Surjection rule} \\

   \textbf{Variant 1}: if we can define a surjective function with domain $A$ and codomain $B$ then
   \[
     |A| \geq |B|
   \] 
   \textbf{Variant 2}: if we can define a function $f: A \rightarrow B$ then 
   \[
     |A| \geq |Ran(f)|
   \] 

   \textbf{Variant 3} If $f: A \rightarrow B$ then $f' : A \rightarrow Ran(f) $ where $f'(x) = f(x) $ is surjective
\end{framed}

\begin{framed}
   \textbf{Injection rule} :

   \textbf{Variant 1} If we ca define an injective function with domain $A$ and codomain $B$ then
   \[
     |A| \leq |B|
   \] 
\end{framed}



